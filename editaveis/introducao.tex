\chapter{Introdução}

Este trabalho caracteriza-se como uma monografia submetida ao curso de graduação em Engenharia de \textit{Software} da Universidade de Brasília, Campus Gama. O trabalho está organizado de forma a prover um bom entendimento ao leitor, apresentando inicialmente uma contextualização sobre o assunto tratado, os objetivos do trabalho, bem como um detalhamento acerca da problemática identificada.

\section{Contextualização}

Nas últimas décadas, o controle de qualidade e uso de padrões atraíram muita atenção. Contudo, historicamente, este assunto é muito antigo \cite{qualidadesoftware}. A exemplo dessa afirmação, tem-se os diversos padrões de medida e controle de produção que foram estabelecidos há mais de quatro mil anos pelas antigas civilizações, como os egípcios.

A partir da observação da linha cronológica de evolução da qualidade, um grande marco, sem dúvida alguma, foi a revolução industrial. Nesse período, a expansão industrial fomentou uma situação de concorrência entre as diversas empresas que surgiram e assim, o processo de melhoria contínua passou a ser buscado. A indústria de desenvolvimento de \textit{software}, embora tenha sido criada em um período mais recente, também está inserida nesse contexto.

As empresas que atuam no desenvolvimento de \textit{software} possuem os desafios de lidar com a crescente taxa de mudanças tecnológicas e adicionalmente, com o aumento dos níveis de concorrência em escala global. Mediante esta perpectiva, a fim de permanecer, as empresas buscam constantemente novas estratégias para se diferenciarem de seus concorrentes.

Nesse âmbito, deve-se considerar também a crescente complexidade dos produtos de \textit{software}. Segundo Sommerville (2003), é válido notar que \textit{software} está presente em várias atividades do cotidiano das pessoas e isso favorece ainda mais a demanda por este produto, acarretando na construção de produtos até mesmo inusitados dependendo da necessidade. Por outro lado, para Dijkstra (1972), o aumento de complexidade deve-se também à rápida melhoria e aperfeiçoamento que se obteve na criação de máquinas.

Como resultado da combinação entre os fatores citados no parágrafo anterior e a crescente pressão imposta pelo mercado, as empresas se voltam para investigação das abordagens de validação e das técnicas de verificação para garantir o desenvolvimento de produtos de valor agregado e de alta qualidade \cite{vbse1}.

Embora na contemporaneidade já se tenha metodologias e práticas para a construção de \textit{software} de forma mais rigorosa e controlada, ainda é possível perceber problemas mencionados na década de 70, sendo eles \cite{qualidadesoftware}:

\begin{itemize}
	\item Cronogramas não observados.
	\item Projetos com tantas dificuldades que são abandonados.
	\item Módulos que não operam corretamente quando combinados.
	\item Programas que não fazem exatamente o que era esperado.
	\item Programas tão difíceis de usar que são descartados.
	\item Programas que simplesmente param de funcionar.
\end{itemize}

Dentre as metodologias concebidas para resolver os problemas enfrentados no desenvolvimento de \textit{software}, tem-se abordagens mais tradicionais (processo unificado por exemplo) e abordagens mais adaptativas (métodos ágeis), sendo estas mais recentes e amplamente adotadas por diversas organizações na atualidade.

Intrínseco à metodologia de desenvolvimento, tem-se as práticas de verificação de \textit{software}. Estas, quando corretamente aplicadas, resolvem boa parte dos problemas identificados no desenvolvimento de \textit{software} citados anteriormente.

Para as empresas fazerem uso conjunto das práticas de verificação e validação de software é necessário que haja uma estratégia de desenvolvimento que favoreça a aplicação de tais práticas. Contudo, durante a concepção desta metodologia de desenvolvimento, muitas organizações não conseguem abstrair as práticas mais básicas e extremamente necessárias para a construção do produto e acabam por negligenciar determinadas atividades. A exemplo dessa afirmação, tem-se a negligência que se pode contemplar em atividades de implementação e execução de testes e também, nas inspeções de código \cite{cemkaner}. Para atender aos prazos, muitas equipes de projeto decidem protelar essas atividades, tornando a qualidade do produto inferior ao que se poderia obter. Adicionalmente, em muitos casos, as atividades técnicas como as citadas anteriormente não estão alinhadas plenamente aos interesses dos clientes, favorecendo a entrega de um produto de menor valor agregado \cite{vbse2}.

Adotar uma metodologia de desenvolvimento perfeitamente adequada para as necessidades da organização e que também atenda de forma concisa aos interesses de todos os envolvidos no projeto não é uma tarefa fácil. Porém, a perspectiva da Engenharia de \textit{Software} Baseada em Valor fornece uma boa maneira de olhar para o processo de desenvolvimento do produto. É válido ressaltar que os envolvidos em um projeto de desenvolvimento de \textit{software} (clientes, analistas de negócio, gerentes de projetos, arquitetos de \textit{software}, desenvolvedores etc) devem possuir um melhor entendimento das implicações provenientes das decisões efetuadas sobre o produto \cite{vbse1}.

\section{Problema}

Na indústria de desenvolvimento de \textit{software}, em geral, há baixo nível de aplicação das principais práticas propostas pela Verificação de \textit{Software} para a construção de produtos de maior qualidade. Na grande maioria dos projetos da indústria de desenvolvimento, essas práticas são negligenciadas, caracterizando-se como atividades de menor importância \cite{cemkaner}.

Adicionalmente, outro quesito que deve ser ressaltado é que as concepções dos clientes de negócio não são concisamente levadas em conta no momento da execução de atividades mais técnicas como as citadas anteriormente. Para exemplificar esta afirmação, basta analisar o que ocorre durante a elaboração e execução de testes para o sistema que está sendo construído. Testes, muitas vezes não estão organizados para maximizar o valor de negócio e também, não estão alinhados com a missão do projeto \cite{vbse2}.

Na busca de possíveis soluções para a problemática destacada, do ponto de vista técnico, já existem práticas propostas pela Verificação de \textit{Software}, tais como a elaboração de testes, inspeção de código etc. Nesse sentido, seria necessário reunir esse conjunto de boas práticas em um único guia. Por outro lado, para lidar com a conciliação dos interesses de todos os envolvidos em um projeto de construção de \textit{software}, tem-se as abordagens da Engenharia de \textit{Software} Baseada em Valor (VBSE - \textit{Value-Based Software Engineer}).

A VBSE traz considerações de valor para o primeiro plano, de modo que as decisões em todos os níveis possam ser otimizadas, para atender ou conciliar os objetivos explícitos das partes interessadas, do marketing pessoal e analistas de negócio aos desenvolvedores, arquitetos e especialistas em qualidade \cite{vbse1}.

Dessa forma, este trabalho apresenta a seguinte questão de pesquisa:

\begin{itemize}
	\item Como utilizar as inspeções e testes unitários, que são práticas complementares da Verificação de \textit{Software}, em conjunto com os conceitos oriundos da VBSE, na avaliação da qualidade de código no contexto do desenvolvimento ágil?
\end{itemize}

Outro quesito que deve ser evidenciado é que devido às limitações de orçamento e prazos, é inviável inspecionar todo o código do \textit{software} ou em alguns casos, obter um percentual de cobertura de testes de 100\% (cem por cento). Nesse contexto, a VBSE se torna ainda mais importante, pois minimamente, as partes mais críticas e que mais agregam valor para o cliente devem ter a qualidade assegurada.

\section{Objetivo}

\subsection{Objetivo Geral}

O objetivo geral deste trabalho é propor um conjunto de atividades, práticas e ferramentas que favoreçam as atividades de inspeção e elaboração de testes unitários, levando em consideração conceitos da VBSE. Por meio deste conjunto, busca-se alcançar bons níveis de manutenibilidade de código no contexto de desenvolvimento ágil.

Este trabalho, como mencionado, possui foco em metodologias ágeis, mais especificamente no \textit{Scrum} e nas práticas mais básicas e complementares da verificação de \textit{software}, sendo elas a implementação de testes unitários e inspeções de código. O \textit{framework} aqui elaborado, reúne essas práticas, a partir de uma ótica de análise de efetividade das mesmas, com atividades do \textit{Scrum}. Naturalmente, exercendo um controle maior sobre a escrita do código que concretiza o \textit{software}, alinhando esta atividade ao propósito do projeto, será possível obter produtos de maior qualidade.

\subsection{Objetivos Específicos}

Por objetivos específicos para este trabalho, tem-se:

\begin{itemize}
	\item Consolidar o entendimento acerca dos elementos necessários à construção do \textit{framework} de avaliação da qualidade de código.
	\item Pesquisar abordagens inerentes às inspeções e aos testes unitários para a elaboração do \textit{framework}.
	\item Definir detalhes acerca da execução das atividades propostas pelo \textit{framework} junto às atividades presentes no \textit{Scrum}.
	\item Incorporar diretrizes propostas pela Engenharia de \textit{Software} Baseada em Valor ao \textit{framework}.
\end{itemize}

\section{Organização do Documento}

\begin{itemize}
	\item \textbf{Capítulo 2 - Referencial Teórico}: apresenta conceitos e abordagens relacionados ao tema deste trabalho, explicitando as informações obtidas a partir da pesquisa bibliográfica e também, a partir da realização da revisão sistemática.

	\item \textbf{Capítulo 3 - Metodologia}: especifica a metodologia adotada para a pesquisa e desenvolvimento deste trabalho, elencando como a efetividade do \textit{framework} será avaliada e também, quais instituições serão utilizadas como caso durante a aplicação do \textit{framework}.

	\item \textbf{Capítulo 4 - \textit{Framework} de Avaliação de Código}: apresenta os \textit{checklists} elaborados, correlacionando estes ao detalhamento do \textit{framework} proposto.

	\item \textbf{Capítulo 5 - Materiais e Métodos}: apresenta uma caracterização detalhada dos projetos adotados para uso do \textit{framework} e descrição de todas as ferramentas de trabalho utilizadas para coleta de informações.

	\item \textbf{Capítulo 6 - Aplicação do \textit{Framework}}: apresenta os resultados obtidos a partir da execução dos ciclos de pesquisa-ação bem como uma análise dos dados coletados.
\end{itemize}